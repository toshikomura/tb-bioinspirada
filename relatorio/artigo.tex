 \documentclass[a4paper,12pt]{article}
 %\documentclass[a4paper,12pt,twocolumn]{article}
 \usepackage{graphicx}    % needed for including graphics e.g. EPS, PS
 \usepackage[brazil]{babel} % needed for including portugues language
 \usepackage[utf8]{inputenc}
 \usepackage[T1]{fontenc}
 \usepackage{hyperref}
 \usepackage{indentfirst}
 \usepackage[margin=0.5in]{geometry}

 \begin{document}
  \title{Ajuste de Pesos de uma Rede Neural utilizando PSO para o Mario AI Competition}	
  \author{
    Fabio Yuki Kubagawa\\ {\small Universidade Federal do Paraná}\\ {\small kyongxd@gmail.com}
    \and
    Gustavo Toshi Komura\\ {\small Universidade Federal do Paraná}\\ {\small gustavo.toshi.k@gmail.com}
  }
  \date{\today}
  \maketitle
  % End of title
  \thispagestyle{empty}
  
  % Start Resumo
  \begin{abstract}

    Este artigo apresenta uma forma de ajuste de pesos de uma rede neural utilizando o PSO 
    para o Mario AI Competition. Neste novo algoritmo são utilizados dois tipos de 
    topologia de vizinhança do PSO o totalmente conectado e o em anel, variando o tamanho 
    da população entre 50, 100 ou 300, e também dois tamanhos de rede neural, a SmallMLP 
    3x3 e MediumMLP 5x5. Os resultados dos experimentos mostraram que uma rede neural com 
    tamanho SmallMLP, população igual a 300 e com uma vizinhança totalmente conectada, possui 
    a melhor fitness com relação às outras múltiplas configurações do novo algoritmo. No 
    entanto esta configuração perde para o AG (Algoritmo Genético), algoritmo que já estava 
    implementado no código fonte do Mario AI Competition. Além disso foi possível observar 
    que o tipo de topologia não faz tanta diferença nos resultados e que com o aumento da 
    população os resultados tendem a melhorar.
  
  \end{abstract}   
  % End Resumo
  
  % Start Introdução
  \section{Introdução}
    
    Jogos digitais possuem ambientes complexos, dinâmicos, porém com um objetivo claro. A criação 
    de um agente inteligente para solucionar um jogo não é trivial. A competição Mario AI 
    Competition, organizada anualmente, foi criada para que a comunidade possa desenvolver, 
    explorar e estudar soluções para um jogo de plataforma, nos quais o agente precisa realizar 
    uma rota até seu objetivo passando por vários obstáculo. Esse tipo de problema pode ser 
    resolvido utilizando Redes Neurais Artificiais, as quais dependem de um algoritmo que 
    faça o ajuste dos seus pesos, que determina a sua saída. Assim, este trabalho tem como 
    objetivo realizar um estudo sobre a utilização do algoritmo Particle Swarm Optimization 
    para o ajuste dos pesos de uma Rede Neural Artificial na resolução do problema do 
    Mario AI Competition.

    Para isso, após a definição de conceitos essenciais dos algoritmos e do problema, serão 
    analisados como o PSO se comporta com diversas configurações de parâmetros e estruturas 
    diferentes. Também iremos analisar como o algoritmo se comporta neste problema utilizando 
    diferentes topologias de vizinhança e diferentes estruturas da rede.

   \subsection{Redes Neurais Artificiais}
      Redes Neurais Artificiais (RNA) são modelos simplificados do sistema nervoso humano. 
      Elas são compostas por diversos neurônios artificiais dispostos em camadas e interligados 
      por conexões geralmente associadas a pesos. Essa técnica é utilizada para solucionar 
      diversos problemas pela sua capacidade de aprender e se adaptar aos dados fornecidos. 
      (KAYARVIZHY, KANMANI e UTHARIARAJ, 2013).

    \subsection{Particle Swarm Optimization}
      A otimização por enxame de partículas (PSO - Particle Swarm Optimization) é uma 
      metaheurística proposta por (KENNEDY e EBERHART, 1995), inspirada no comportamento 
      social de uma revoada de pássaros, insetos ou de um cardume de peixes. A idéia por trás 
      dessa metaheurística é a de que os indivíduos dentro de um grupo ou de uma sociedade se 
      baseiam em experiências bem sucedidas de si mesmo e dos outros integrantes quando vão 
      tomar decisões (MAIA, 2009).

    \subsection{Ajuste de pesos numa Rede Neural Artificial utilizando PSO}
      Numa RNA, um dos critérios mais importantes é o ajuste de pesos. Uma das formas de 
      calcular esses pesos, entre várias outras, é adaptando o PSO. Utilizamos todas as 
      combinações de valores possíveis para cada peso como nosso espaço de busca, cada 
      partícula voa sobre esse espaço de busca tentando encontrar o conjunto ótimo de pesos. 
      Para avaliarmos a fitness duma partícula do PSO, utilizamos a precisão que os pesos da 
      partícula têm ao serem aplicados à RNA. (KAYARVIZHY, KANMANI e UTHARIARAJ, 2013).
    
  % End Introdução
  
  % Start Problema e Implementação
  \section{Problema e Implementação}
      
    \subsection{Mario AI Competition}
    
      A Mario AI Competition é uma competição realizada anualmente desde 2009 para o 
      desenvolvimento e aprendizado do melhor agente para uma versão de Super Mario Bros. 
      \url{http://julian.togelius.com/mariocompetition2009/}
      A competição é baseada no Infinite Mario Bros por Markus Persson 
      \url{https://mojang.com/notch/mario/}, uma cópia do Super Mario Bros open source 
      desenvolvida em Java com geração de fases procedural. Os autores da competição criaram 
      uma interface para controladores e possibilitaram rodar a aplicação sem a interface 
      gráfica, para possibilitar testes e coleta de dados.
    
      Nesta interface que os agentes possuem, a cada passo de tempo, que ocorre 24 vezes 
      por segundo, o agente recebe uma representação do ambiente, tanto as informações de 
      blocos ou inimigos que estão na tela quanto a posição exata do agente e onde ele está 
      no seu pulo. Dado essa representação, o agente deve devolver uma ação, composta de 5 
      bits, que representam os 5 botões: esquerda, direita, baixo, pulo e correr.
      Um agente pode ser avaliado em relação ao seu desempenho por diversos fatores, mas o 
      utilizado para a competição foi o quão longe os agentes chegavam em cada fase.
      
      O framework disponibilizado para a competição possui toda a estrutura necessária para 
      o desenvolvimento de novos agentes, além de incluir uma base inicial de alguns agentes 
      simples, incluindo um agente que utiliza RNA. Utilizando esse framework também podemos 
      criar instâncias do problema para os agentes resolverem, que é gerado a partir dos 
      parâmetros fornecidos, como dificuldade da fase, tamanho, tempo limite, inimigos 
      presentes, entre outros.
    
      A competição apresenta um problema dinâmico, exigindo que os agentes consigam 
      interpretar um espaço de estados muito grande e consigam reagir a situações bem diferentes.
      
      \begin{figure}[!htb]
	\centering
	\includegraphics[scale=0.5]{mario}
	\caption{Visualização gráfica do Mario AI Competition}
	\label{Figura 1}
      \end{figure}

    \subsection{Redes Neurais para resolver o problema do Mario AI}
    
      A implementação encontrada para resolver este problema que utiliza RNAs utiliza a idéia 
      de utilizar matrizes para representar o ambiente próximo ao agente e algumas variáveis 
      para melhor descrever o estado que o agente está como entrada e um vetor binário de 5 
      posições que representa as 5 ações possíveis que o agente pode tomar.
      Existem três implementações com tamanhos de rede diferentes. Na rede Small utilizamos 
      duas matrizes 3x3, uma para representar as colisões, com 0 para espaço vazio e 1 para 
      espaço sólido, e outro para representar a existência de inimigos, com 0 para espaço vazio 
      e 1 para espaço ocupado por um inimigo, com o centro da matriz sendo a posição atual do 
      agente. Na rede Medium utilizamos duas matrizes semelhantes 5x5, e na Large elas são 7x7. 
      Além dessas duas matrizes, todas as implementações também utilizam uma entrada que indica 
      se o agente está pisando no chão e outra que indica se ele pode ou não iniciar um pulo de 
      onde ele está.
    
      Essas RNAs possuem uma camada escondida com 10 neurônios entre a camada de entrada e de 
      saída. Para cada conexão entre cada entrada e cada neurônio da camada escondida e entre cada 
      neurônio da camada escondida para cada saída, temos um peso.
      
      O algoritmo utilizado para ajuste desses pesos é, por padrão, um algoritmo genético. Esse 
      algoritmo genético possui 100 indivíduos que armazenam todos os pesos da RNA e passam por 100 
      gerações. A cada geração é feito uma avaliação de todos os indivíduos da população, então é 
      realizado um elitismo de 50\% dessa população. Essa elite é então copiada para a outra 
      metade e é feito uma mutação sobre cada uma delas. Essa mutação é feita da seguinte forma: 
      cada um dos pesos do indivíduo é somado com um valor aleatório gerado por uma função que 
      segue uma função de distribuição normal com média igual a zero e com 70\% das saídas 
      entre -1 e 1, esse valor aleatório é multiplicado por uma constante de mutação definida 
      em 0,1.
    
      Esta implementação consegue atingir resultados satisfatórios, mas ainda menor do que os 
      agentes baseados em regras simples, pelo fato de terem um comportamento mais 
      determinístico e não dependerem da fase de treinamento, só seguirem uma heurística.

  % End Problema e Implementação
  
  % Start Experimento
  \section{Experimento}
      
      O foco dos experimentos é fazer a comparação do novo algoritmos com suas múltiplas 
      configurações, e além de compará-los entre si, fazer uma comparação com o AG 
      (Algoritmo Genético) que já estava implementado no código fonte do Mario IA Competition.

      Para a estrutura do PSO, escolhemos como critério de parada do algoritmo uma evolução máxima 
      de 100 gerações da população ou o término da fase, onde neste caso é uma fitness com valor 
      igual a 4096. Para o tamanho da população, inciamos os testes com 50 partículas, sendo em 
      seguida trocada por 100 partículas e por fim para 300 partículas. Por fim escolhemos dois 
      tipos de topologias, a vizinhança totalmente conectada (TC), onde uma partícula está conectada 
      com todas as outras partículas da população, e a vizinhança em anel (Anl), onde pré-definimos 
      dois vizinhos para cada partícula, que não necessariamente são vizinhos geográficos.

      Para a rede neural utilizamos dois tipos de tamanhos distintos, que neste caso são a SmallMPL 
      3x3 e a MediumMPL 5x5, onde somente para recapitular, uma matriz 3x3 representa o ambiente 
      envolta do personagem, com duas matrizes, uma de colisão e uma outra de inimigos. Acabamos 
      por não inserir o tamanho LargeMPL, pois ao se fazer os testes com a SmallMPL., percebemos 
      que os testes com ele acabariam sendo muito custosos e pelo tempo que nos restava isso não 
      seria possível.

      Para o algoritmo AG (Algoritmo Genético), utilizamos o algoritmo que já havia sido 
      implementado e que estava contido no código fonte do Mario AI Competition. Esse algoritmo 
      genético possui como condição de parada uma evolução máxima de 100 gerações. Além disso 
      possui 100 indivíduos no tamanho da população com um elitismo de 50\% e uma constante de 
      mutação definida em 0,1.
      
      Para os experimentos foi utilizado o ambiente de desenvolvimento intelliJ IDEA no ambiente 
      Linux. O código foi modificado a partir da fonte disponível em 
      \url{https://code.google.com/p/marioai/}. Rodamos três vezes o processo de treinamento de 
      rede para cada combinação de dificuldade, vizinhança e estrutura de rede, realizando 
      uma média aritmética do quão longe o agente chegou em cada treinamento, a qual foi a 
      métrica utilizada para avaliar cada algoritmo. Cada fase tem 4096 unidades de comprimento 
      e portanto é a pontuação máxima que um algoritmo pode obter.

      As fases, apesar de geradas aleatoriamente, são criadas a partir de uma semente, que foi 
      definida em 0 para que todo agente treinado na mesma dificuldade seja treinado na exata 
      mesma fase, para uma comparação mais justa.
  
  % Start Análise dos Resultados
  \section{Análise dos Resultados}
  
    Os resultados gerais dos experimentos podem ser vistos na Figura 2, temos uma tabela com as três 
    primeiras colunas em azul claro indicando qual rede foi utilizada, com tamanho da rede, Small 3x3 ou 
    Medium 5x5, a topologia de vizinhança utilizada, e o tamanho da população. Em seguida 
    temos as 11 colunas que representam cada uma das dificuldades diferentes de fases nas quais 
    os algoritmos foram testados. Nas células destas colunas podemos notar os maiores valores 
    de cada coluna em amarelo, os segundos melhores valores em cinza e os piores valores em vermelho.
    
    Dentre as topologias de vizinhança, representamos a Totalmente Conectada como TC e a em Anel por Anl.
    
    \begin{figure}[!htb]
	\centering
	\includegraphics[scale=0.6]{tabela_completa}
	\caption{Resultados dos experimentos}
	\label{Figura 2}
      \end{figure}
    
    É aparente como a implementação padrão do algoritmo genético (AG) possuiu um desempenho melhor. 
    Isto aparenta vir do fato do processo de mutação utilizado ser mais drástico, trazendo mais 
    diversidade, e tendo uma população e gerações suficientes para alcançar um bom resultado.
    
    Podemos notar também que, consistentemente, os resultados os algoritmos que utilizam PSO têm 
    resultados melhores quando o tamanho da sua população é maior. Com base nesse fato iremos 
    basear nossas análises sobre os resultados com população igual a 300.
    
    Após o AG, o algoritmo com PSO que utiliza a estrutura Small, topologia Totalmente Conectado 
    possuiu um desempenho no geral melhor que os outros, mesmo que o mesmo algoritmo com 
    populações mais baixas acumulou o maior número de piores resultados. Assim percebemos que 
    esse tipo de estrutura, mais reativa, pois tem uma visão mais “fechada” do ambiente, 
    necessita duma população maior que os outros para obter os mesmos resultados.
    
    Uma anomalia que pode-se notar na Figura 2 é como várias vezes os algoritmos possuem 
    resultados melhores numa fase de dificuldade maior em relação à outra com dificuldade 
    menor. Como as fases são geradas aleatoriamente, isso pode acabar acontecendo onde 
    uma fase não fica na prática mais difícil apesar dos parâmetros utilizados para gerá-la 
    tiverem um potencial de deixá-la mais difícil.
      
    \subsubsection{Impacto da Vizinhança}
    
      Pela Figura 3 podemos visualizar uma análise direcionada à topologia de vizinhança. 
      As 2 sub-tabelas mais a esquerda indicam na primeira coluna a dificuldade e nas duas 
      seguintes os resultados de um determinado algoritmo em cada uma dessas dificuldades, 
      e os quadradinhos que estão em cinza possuem o melhor resultado da comparação entre 
      estas duas colunas. A última linha (“Qtd”) indica a contabilização da quantidade de 
      melhores resultados de cada algoritmo. A tabela mais a direita indica o soma de 
      (“Qtd”) de TC e Anl da quantidade de melhores resultados de cada sub-tabela.

      \begin{figure}[!htb]
	\centering
	\includegraphics[scale=0.5]{tabela_topologia}
	\caption{Análise dos resultados com relação à topologia de vizinhança}
	\label{Figura 3}
      \end{figure}
      
      Pela Figura 3 podemos notar que a topologia não parece ter relação com a estrutura 
      da rede, já que as duas primeiras sub-tabelas possuem subtotais iguais.
      
      Também analisamos que a quantidade de vezes que a topologia em Anel foi melhor que 
      a Totalmente Conectada é bem próxima da Totalmente Conectada, apesar de não superá-la. 
      Por essa diferença ser tão pequena, e muitas vezes os resultados das duas ser 
      bem próximo, concluímos que a topologia não faz uma diferença significativa o 
      suficiente neste caso, onde usamos 100 gerações.

    \subsubsection{Impacto da estrutura RNA}
    
      Também podemos visualizar os resultados focando na estrutura da rede utilizada. 
      Na Figura 4 temos essa comparação, com as diferentes estruturas de rede sendo 
      comparadas, contando quantas vezes o desempenho da estrutura Small ou Medium 
      supera a outra. Na primeira coluna da tabela na Figura 4 temos as dificuldades seguido 
      por 4 colunas que determinam os resultados em cada algoritmo, e os quadradinhos 
      que estão em cinza possuem o melhor resultado da comparação entre estas quatro 
      colunas. A última linha (“Qtd”) indica a contabilização da quantidade de melhores 
      resultados de cada algoritmo.

      \begin{figure}[!htb]
	\centering
	\includegraphics[scale=0.6]{tabela_estrutura}
	\caption{Análise dos resultados com relação à estrutura da rede}
	\label{Figura 4}
      \end{figure}
      
      Na Figura 4 podemos ver como a diferença entre a estrutura Small e Medium é 
      significativa, com 8 vezes o resultado da Small sendo superior à Medium.
      
      A principal diferença entre as duas estruturas é que a Small possui uma 
      visibilidade mais baixa que a Medium. Isso faz com que os agentes que 
      utilizam a estrutura Small sejam mais reativos a situação, só percebendo 
      inimigos quando eles estão bem próximos dele, ou só notando buracos e obstáculos 
      quando ele já está do lado dele. No Medium, como ele tem um campo de visão maior 
      e não consegue diferenciar espaços em relação a posição geográfica deles, ou 
      seja, diferenciar o que está atrás do que está na frente, muitas vezes eles 
      acabam sendo influenciados por objetos que não deveriam modificar sua saída. 
      Com 100 gerações talvez não há tempo suficiente para isso ser aprimorado, porque 
      existem muitas combinações de entradas que são bem diferentes mas que deveriam 
      ter saídas iguais.

    \subsection{Considerações Finais}
      Apesar dos algoritmos propostos não tiverem um desempenho melhor que o algoritmo 
      padrão já implementado, muitas informações novas podem ser extraídas dos testes 
      realizados. Notamos que uma das principais fraquezas que todos os algoritmos 
      analisados apresentam é sua saída. A definição de como a saída e a entrada da 
      rede é feita é tão importante quanto à estrutura de seus neurônios, e a que foi 
      utilizada aparenta não ser a ideal.
    
      Um dos maiores problemas encontrados durante a inspeção de certos testes no modo 
      gráfico foi a presença, repetidamente, de situações onde o agente possui como 
      saída uma combinação de ações indesejada, como andar para a esquerda e para a 
      direita simultaneamente, resultando no agente ficando preso. Como o número de 
      saídas possíveis que isso acontece não é baixa, isso acabava sendo comum, 
      principalmente nas primeiras gerações que possuem um pouco da influência da 
      sua geração aleatória.
    
      Uma possível solução para esse problema é, ao invés de utilizarmos como saída 
      o vetor de bits que será passado para o agente, codificarmos de alguma forma 
      uma lista de ações possíveis que o agente pode tomar, e depois transformar 
      essa ação num vetor de bits.

      \begin{figure}[!htb]
	\centering
	\includegraphics[scale=0.4]{grafico_comparacao}
	\caption{Gráfico de fitness das topologias em cada dificuldade}
	\label{Figura 5}
      \end{figure}
      
      Na Figura 5 podemos ver como apesar do desempenho das outras topologias 
      for, em geral, inferior à AG, elas não estão tão longe dela, mas tendem a 
      “flutuar” bastante de valor. A discrepância grande de valores se deve ao fato da 
      fitness ser o quão longe o agente chegou na fase de 4096 de comprimento. 
      Geralmente existem certos obstáculos mais significativos que exigem mais do 
      agente em certos pontos, de forma que passando de um deles aumenta a fitness 
      consideravelmente.
    
    % End Análise dos Resultados
  
  % Start Conclusão
  \section{Conclusão}

    Este trabalho apresentou uma introdução sobre Redes Neurais e o algoritmo PSO 
    que podem ser utilizados em conjunto para resolver diversos problemas. Então 
    apresentamos como o problema do Mario AI Competition pode ser solucionado 
    utilizando essas ferramentas. Com a definição do problema e a base já 
    implementada, diversos parâmetros e modelos foram estudados, implementados e 
    seus resultados analisados, afim de comparar o desempenho do algoritmo PSO com 
    suas diversas configurações com o algoritmo genético já implementado.

    Os resultados dos testes nos mostraram que a topologia é irrelevante com relação a 
    fitness escolhida e que o aumento da quantidade de partículas influência diretamente 
    e positivamente nela. Também foi possível observar que o AG (Algoritmo Genético) é 
    melhor que novo algoritmo com ajuste de pesos com o PSO, no entanto numa comparação 
    entre eles, percebeu-se que os resultados ficaram bem próximos, apesar dos 
    resultados da configuração do PSO flutuarem bastante.

    Para trabalhos futuros, modificar a estrutura de saída da rede, para evitarmos saídas 
    redundantes, ou expandir a análise dos parâmetros podem prover resultados melhores 
    para o problema. Testes podem ser realizados explorando um número maior de partículas, 
    para analisar se há decaimento do desempenho do algoritmo depois de um certo limiar, 
    aumentando o número de gerações, notando se a topologia em anel possa ter um desempenho 
    superior ao totalmente conectado, e realizando uma quantidade significativamente maior 
    de casos de teste, para obtermos resultados mais estatisticamente corretos.

    O problema do Mario AI Competition é complexo, exigindo que os algoritmos que o 
    resolvam sejam genéricos o suficientes para conseguir resolver as diversas situações 
    possíveis. É um problema dinâmico que testa o algoritmo em ângulos interessantes e pode 
    ser uma boa forma de comparar algoritmos bioinspirados para esses tipos de situações.

  % End Conclusão
  
  % Start Referências  
  \begin{thebibliography}{9}

    \bibitem{kinebe95}
      Kennedy J., Eberhart, R.,
      Particle Swarm Optimization,
      ( Proceedings of IEEE International Conference on Neural Networks,
      1995 ).

    \bibitem{mai09}
      Maia, P. P. C.,
      Uma Metaheurística Híbrida Paralela para o Problema do Caixeiro Viajante,
      Universidade Federal do Rio Grande do Norte,
      Natal/RN,
      2009.

    \bibitem{kaykanuth13}
      Kayarvizhy, N., Kanmani, S., Uthariaraj, V. R.,
      Improving Fault Prediction using ANN-PSO in Object Oriented Systems,
      ( International Journal of Computer Applications,
      2013 ).

    \bibitem{zyuzhoche10}
      Ziyu Chen, Zhongshi He e Cheng Zhang,
      Chongqing University, Chongqing, China,
      Particle swarm optimizer with self-adjusting neighborhoods
      ( GECCO '10 Proceedings of the 12th annual conference on Genetic and evolutionary computation,
      pg.9-14,
      ACM New York, NY, USA ©2010,
      2010 ).
    
    \bibitem{xin99}
      Xin Yao,
      Sch. of Comput. Sci., Birmingham Univ., UK,
      Evolving artificial neural networks
      ( Proceedings of the IEEE  (Volume:87 ,  Issue: 9, pg.1423 - 1447),
      Sep 1999 ).
      
    \bibitem{kendavamb09}
      Kenneth O. Stanley, David B. D'Ambrosio e Jason Gauci,
      A hypercube-based encoding for evolving large-scale neural networks
      ( Journal Artificial Life archive,
      Volume 15 Issue 2, Spring 2009,
      pg.185-212,
      MIT Press Cambridge, MA, USA,
      2009 ).
      
    \bibitem{julserjanjur09}
      Julian Togelius 	IT University of Copenhagen, Copenhagen S, Denmark
      Sergey Karakovskiy 	IDSIA, Manno-Lugano, Switzerland
      Jan Koutník 	IDSIA, Manno-Lugano, Switzerland
      Jürgen Schmidhuber 	IDSIA, Manno-Lugano, Switzerland,
      Super mario evolution
      ( Proceeding CIG'09 Proceedings of the 5th international conference on Computational Intelligence and Games
      pg.156-161
      IEEE Press Piscataway, NJ, USA ©2009 
      2009 ).
      
    \bibitem{zdejanmir09}
      Zdeněk Buk, Jan Koutník e Miroslav Šnorek,
      NEAT in HyperNEAT Substituted with Genetic Programming
      ( Springer Berlin Heidelberg,
      pg.243-252,
      2009 ).

  \end{thebibliography}
  % End Referências

 \end{document}